% !TeX root = ../document.tex

\section{Introduction}

The basic definition of an urban heat island (UHI) effect is \ldq{}[...] that an urban area or metropolitan area is significantly warmer than its surrounding rural areas due to human activities\rdq{}\cite{takebayashi_chapter_2020}. Higher temperatures than those in the surrounding area can indicate heat islands. As known, infrastructure such as buildings, roads, etc. absorb and re-emit the sun’s radiation in the form of heat, whereas natural landscapes such as forests and water bodies have a cooling effect \cite{us_epa_learn_2014}.

Thus, replacing natural areas with dense concentrations of buildings, infrastructure, pavements and other sealed surfaces that absorb and retain heat, leads to a localized increase in air temperature. 

Increased heat islands within the city have a negative effect, not only on ecosystems, but on people’s health and well-being especially for the vulnerable population. Therefore, city planners should pay attention to the impact of certain design choices to the urban climate.

\subsection{Project Goals}
The goal of our study is to analyze temperature data in order to be able to create a visual product of the urban heat effect. This, subsequently can be used to support planning authorities by making this effect visible. Current developments in regard to climate change urge the implementation of countermeasure and adaptation strategies in urban areas, where more greening is obviously needed. \cite{ketterer_comparison_2015}

% TODO: more about background

Our approach includes the comparison of changes in temperature over the course of the warmest day in 2019 and 2020 in Berlin. What study area did you choose and why?
The study area includes the capital city Berlin and the surrounding area. This site was chosen due to the availability of various data points and measurements, and furthermore, we hope to find a more significant local heat island effect, as compared to other areas in Germany. Although the effect should be identifiable in similar environments, the comparison of different interpolation methods stands in the focus of this project, which is why we decided to use a more self-evident study site. Thus, the result of the comparison of the different methods below should point towards one overall method, which is best-suitable for answering questions in relation to the urban heat island effect.

\subsection{Data: Selection and preparation (workflow)}