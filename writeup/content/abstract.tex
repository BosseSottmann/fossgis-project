% !TeX root = ../document.tex

\noindent\textbf{Abstract.} The heat island effect describes the phenomenon that urban areas experience higher heat accumulation over the course of a day than rural areas and natural landscapes due to the absorption and re-emittance of infrastructure such as buildings, roads and other sealed surfaces.

Increased heat islands within the city have a negative effect, not only on ecosystems, but on people’s health and well-being especially for the vulnerable populations e.g. infants and elderly people. Hence, the heat island affect should be considered during city planing activities.

The goal for this project is to analyze temperature data with Berlin to identify possible locations of heat islands and their development during the day. The raw data was taken from the citizen science project openSenseMap\footnote{\url{https://opensensemap.org/}}, aggregated into ten minute average measurements and used to interpolate a full map of Berlin using the stations. Due to the incompleteness of measurements the stations were filtered.
